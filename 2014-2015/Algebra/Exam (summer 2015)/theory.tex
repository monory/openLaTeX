\documentclass[a4paper]{article}
\usepackage[utf8]{inputenc}
\usepackage[russian]{babel}
\usepackage{amsmath}
\usepackage{amssymb}
\usepackage{amsthm}
\usepackage{mathtext}
\usepackage{mathtools}
\usepackage{microtype}
\usepackage[left=2cm,right=2cm,top=2cm,bottom=2cm]{geometry}

\theoremstyle{plain}
\newtheorem*{theorem}{Теорема}
\newtheorem{proposal}{Предложение}
\theoremstyle{definition}
\newtheorem{definition}{Определение}
\numberwithin{definition}{section}
\numberwithin{proposal}{section}

\newcommand{\im}{\mathrm{im}}
\newcommand{\rk}{\mathrm{rk}}

\everymath{\displaystyle}
\begin{document}
\sloppy
\author{Чудинов Никита (группа 14-4)}
\date{}
\title{\vspace{-1.7cm}Краткая теория к экзамену по высшей алгебре 19-20 июня 2015}
\frenchspacing

\maketitle

\section{Лекция 1}

\begin{definition}
\emph{Множество с бинарной операцией} --- это множество \(M\) с заданным отображением
\begin{equation*}
	M \times M \rightarrow M, \quad (a, b) \mapsto a \circ b.
\end{equation*}
\end{definition}

\begin{definition}
Множество с бинарной операцией \((M, \circ)\) называется \emph{полугруппой}, если данная бинарная операция ассоциативна, то есть
\begin{equation*}
	a \circ (b \circ c) = (a \circ b) \circ c, \forall a,b,c \in M.
\end{equation*}
\end{definition}

\begin{definition}
Полугруппа \((S, \circ)\) называется \emph{моноидом}, если в ней есть \emph{нейтральный элемент} \(e \in S\), такой, что
\begin{equation*}
	\forall a \in S: \quad e \circ a = a \circ e = a.
\end{equation*}
\end{definition}

\begin{definition}
Моноид называется \emph{группой}, если для каждого элемента \(a \in S\) найдётся \emph{обратный элемент} \(a^{-1}\), такой, что
\begin{equation*}
	a \circ a^{-1} = a^{-1} \circ a = e.
\end{equation*}
\end{definition}

\begin{definition}
Группа \(G\) называется \emph{коммутативной} или \emph{абелевой}, если групповая операция коммутативна, то есть
\begin{equation*}
	a \circ b = b \circ a, \forall a, b \in G.
\end{equation*}
\end{definition}

\begin{definition}
\emph{Порядок} группы \(G\) --- это количество элементов в \(G\). Группа называется конечной, если её порядок конечен, и бесконечной иначе. Обозначается \(|G|\).
\end{definition}

\begin{definition}
Подмножество \(H\) группы \(G\) называется \emph{подгруппой}, если
\begin{equation*}
 	|H| > 0, \quad a \circ b^{-1} \in H, \forall a,b \in H.
\end{equation*}
\end{definition}

\begin{proposal}
Всякая подгруппа в \((\mathbb{Z}, +)\) имеет вид \(k\mathbb{Z}\) для некоторого \(k \in \mathbb{N}\)
\end{proposal}

\begin{definition}
Пусть \(G\) --- группа и \(g \in G\). \emph{Циклической подгруппой}, порождённой элементом \(g\) называется подмножество
\begin{equation*}
	\{g^n \;|\; n \in \mathbb{Z}\} \in G.
\end{equation*}
Элемент \(g\) называется \emph{порождающим} или \emph{образующим} для этой группы \(\langle g \rangle\).
\end{definition}

\begin{definition}
Пусть \(G\) --- группа и \(g \in G\). \emph{Порядком} элемента \(g\) называется такое наименьшее натуральное число \(m\), что \(g^m = e\). Если такого натурального числа не существует, то говорят, что порядок \(g\) равен бесконечности. Обозначение: \(\mathrm{ord}(g)\).
\end{definition}

\begin{proposal}
Пусть \(G\) --- группа и \(g \in G\). Тогда \(\mathrm{ord}(g) = |\langle g \rangle|\).
\end{proposal}

\begin{definition}
Группа \(G\) называется \emph{циклической}, если 
\begin{equation*}
	\exists \; g \in G, \quad G = \langle g \rangle.
\end{equation*}
\end{definition}

\begin{definition}
Пусть \(G\) --- группа, \(H \subseteq G\) --- подгруппа и \(g \in G\). \emph{Левым смежным классом} элемента \(g\) группы \(G\) по подгруппе \(H\) называется подмножество
\begin{equation*}
	gH = \{gh \;|\; h \in H\}.
\end{equation*}
\end{definition}

\begin{definition}
Пусть \(G\) --- группа, \(H \subseteq G\) --- подгруппа. \emph{Индексом} подгруппы \(H\) в группе \(G\) называется число левых смежных классов \(G\) по \(H\). Обозначается \([G \::\: H]\).
\end{definition}

\begin{theorem}[Лагранж]
Пусть \(G\) --- конечная группа и \(H \subseteq G\) --- подгруппа. Тогда
\begin{equation*}
	|G| = |H| \cdot [G \::\: H].
\end{equation*}
\end{theorem}


\section{Лекция 2}

\begin{definition}
Подгруппа \(H\) группы \(G\) называется \emph{нормальной}, если
\begin{equation*}
 	gH = Hg, \forall g \in G.
\end{equation*} 
\end{definition}

\begin{proposal}
Для подгруппы \(H \subseteq G\) следующие условия эквивалентны:
\begin{enumerate}
	\item \(H\) нормальна;
	\item \(gHg^{-1} \subseteq H, \forall g \in G\);
	\item \(gHg^{-1} = H, \forall g \in G\);
\end{enumerate}
\end{proposal}

\begin{definition}
Множество \(G / H\) с указанной операцией называется \emph{факторгруппой} группы \(G\) по нормальной подгруппе \(H\).
\end{definition}

\begin{definition}
Пусть \(G\) и \(F\) --- группы. Отображение \(\varphi: G \rightarrow F\) называется \emph{гомоморфизмом}, если
\begin{equation*}
	\varphi(ab) = \varphi(a)\varphi(b), \forall a, b \in G.
\end{equation*}
\end{definition}

\begin{definition}
Гомоморфизм групп \(\varphi: G~\rightarrow~F\) называется \emph{изоморфизмом}, если отображение \(\varphi\) биективно.
\end{definition}

\begin{definition}
Группы \(G\) и \(F\) называют \emph{изоморфными}, если между ними есть изоморфизм. Обозначение: \(G \cong F\).
\end{definition}

\begin{theorem}
Всякая бесконечная циклическая группа \(G\) изоморфна группе \((\mathbb{Z}, +)\).
\end{theorem}

\begin{theorem}
Всякая циклическая группа порядка \(n\) изоморфна группе \((\mathbb{Z}_n, +)\).
\end{theorem}

\begin{definition}
С каждым гомоморфизмом групп \(\varphi: G \rightarrow F\) связаны его \emph{ядро}
\begin{equation*}
	\ker(\varphi) = \{g \in G \;|\; \varphi(g) = e_F\}
\end{equation*}
и \emph{образ}
\begin{equation*}
	\im(\varphi) = \{a \in F \;|\; \exists g \in G: \varphi(g) = a\}.
\end{equation*}
\end{definition}

\begin{proposal}
Пусть \(\varphi: G \rightarrow F\) --- гомоморфизм групп. Тогда подгруппа \(\ker(\varphi)\) нормальна в \(G\).
\end{proposal}

\begin{theorem}[о гомоморфизме]
Пусть \(\varphi: G \rightarrow F\) --- гомоморфизм групп. Тогда группа \(\im(\varphi)\) изоморфна факторгруппе \(G/\ker(\varphi)\).
\end{theorem}

\begin{definition}
\emph{Центр} группы \(G\) --- это подмножество
\begin{equation*}
	Z(G) = \{a \in G \;|\; ab = ba, \forall b \in G\}.
\end{equation*}
\end{definition}

\begin{proposal}
Центр \(Z(G)\) является нормальной подгруппой группы \(G\).
\end{proposal}

\begin{definition}
\emph{Прямым произведением} групп \(G_1, \dots, G_m\) называется множество
\begin{equation*}
	G_1 \times \dots \times G_m = \{(g_1, \dots, g_m) \;|\; g_1 \in G_1, \dots, g_m \in G_m\}.
\end{equation*}
\end{definition}

\begin{theorem}[о факторизации и сомножителях]
Пусть \(H_1, \dots, H_m\) --- нормальные подгруппы в группах \(G_1, \dots, G_m\) соответственно. Тогда \(H_1 \times \cdot \times H_m\) --- нормальная подгруппа в \(G_1 \times \dots \times G_m\) и имеет место изоморфизм групп
\begin{equation*}
	(G_1 \times \dots \times G_m)/(H_1 \times \dots \times H_m) \cong G_1/H_1 \times \dots \times G_m/H_m
\end{equation*}
\end{theorem}

\begin{theorem}
Пусть \(n = ml\) --- разложение натурального числа \(n\) на два взаимно простых множителя. Тогда имеет место изоморфизм групп
\begin{equation*}
	\mathbb{Z}_n \cong \mathbb{Z}_m \times \mathbb{Z}_l.
\end{equation*}
\end{theorem}

\section{Лекция 3}

\begin{definition}
Абелева группа \(A\) называется \emph{конечно порождённой}, если найдутся такие элементы \(a_1, \dots, a_n \in A\), что всякий элемент \(a \in A\) представим в виде \(a = s_1a_1 + \dots + s_na_n\) для некоторых целых чисел \(s_1, \dots, s_n\). При этом элементы \(a_1, \dots, a_n\) называются \emph{порождающими} или \emph{образующими} группы \(A\).
\end{definition}

\begin{definition}
Конечно порождённая абелева группа \(A\) называется \emph{свободной}, если в ней существует \emph{базис}, то есть такой набор элементов \(a_1, \dots, a_n\), что каждый элемент \(a \in A\) единственным образом представим в виде \(a = s_1a_1 + \dots + s_na_n\), где \(s_1, \dots, s_n \in \mathbb{Z}\). При этом число \(n\) называется рангом свободной абелевой группы \(A\) и обозначается \(\rk A\).
\end{definition}

\begin{proposal}
Любые два базиса свободной группы содержат одинаковое число элементов.
\end{proposal}

\begin{proposal}
Всякая свободная абелева группа ранга \(n\) изоморфна группе \(\mathbb{Z}^n\).
\end{proposal}

\begin{theorem}
Всякая подгруппа \(N\) свободной абелевой группы \(L\) ранга \(n\) является свободной абелевой группой ранга \(\leqslant n\).
\end{theorem}

\begin{theorem}[о согласованных базисах]
Для всякой подгруппы \(N\) свободной абелевой группы \(L\) ранга \(n\) найдётся такой базис \(e_1, \dots, e_n\) группы \(L\) и такие натуральные числа \(u_1, \dots, u_m, m \leqslant n\), что \(u_1e_1, \dots, u_me_m\) --- базис группы \(N\) и \(u_i | u_{i+1}\) при \(i = 1, \dots, m-1\).
\end{theorem}

\begin{proposal}
Пусть \(e'_1, \dots, e'_n\) --- некоторый набор элементов из \(\mathbb{Z}^n\). Выразив эти элементы через стандартный базис \(e_1, \dots, e_n\), мы можем записать
\begin{equation*}
	(e_1', \dots, e_n') = (e_1, \dots, e_n)C,
\end{equation*}
где \(C\) --- целочисленная квадратная матрица порядка \(n\).

Элементы \(e'_1, \dots, e'_n\) составляют базис группы \(\mathbb{Z}^n\) тогда и только тогда, когда \(\det C = \pm 1\).
\end{proposal}

\begin{definition}
\emph{Целочисленными элементарными преобразованиями строк} матрицы называются преобразования следующих трёх типов:
\begin{enumerate}
	\item Прибавление к одной строке другой, умноженной на целое число;
	\item Перестановка двух строк;
	\item Умножение одной строки на \(-1\).
\end{enumerate}
Аналогично определяются \emph{целочисленные элементарные преобразования столбцов} матрицы.
\end{definition}

\begin{proposal}
Всякую прямоугольную матрицу \(C = (c_{ij}\) размера \(n \times m\) назовём диагональной и обозначим \(\textrm{diag}(u_1, \dots, u_p)\), если \(c_{ij} = 0\) при \(i \neq j\) и \(c_{ii} = u_i\) при \(i = 1, \dots, p\), где \(p = \textrm{min}(n, m)\). 
\end{proposal}


\section{Лекция 4}

\begin{definition}
Конечная абелева группа называется \emph{примарной}, если её порядок равен \(p^k\) для некоторого простого числа \(p\).
\end{definition}

\begin{theorem}
Всякая конечно порождённая абелева группа \(A\) разлагается в прямую сумму примарных и бесконечных циклических подгрупп, то есть
\begin{equation*}
	A \cong \mathbb{Z}_{p_1^{k_1}} \oplus \dots \oplus \mathbb{Z} \oplus \dots \oplus \mathbb{Z},
\end{equation*}
где \(p_1, \dots, p_s\) --- простые числа (необязательно попарно различные), а \(k_1, \dots, k_s \in \mathbb{N}\). Кроме того, число бесконечных циклических слагаемых, а также число и порядки примарных циклических слагаемых определено однозначно. 
\end{theorem}

\begin{proposal}
Всякая конечная абелева группа разлагается в прямую сумму примарных циклических подгрупп, причем число и порядки примарных циклических слагаемых определено однозначно.
\end{proposal}

\begin{definition}
\emph{Экспонентой} конечной абелевой группы \(A\) называется число \(\mathrm{exp} A\), равное наименьшему общему кратному порядков элементов из \(A\).
\end{definition}

\begin{proposal}
Конечная абелева группа \(A\) является циклической тогда и только тогда, когда \(\mathrm{exp} A = |A|\).
\end{proposal}


\section{Лекция 5}

\begin{definition}
\emph{Действием} группы \(G\) на множестве \(X\) называется отображение \(G \times X \rightarrow X, (g, x) \mapsto gx\), удовлетворяющее следующим условиям:
\begin{enumerate}
	\item \(ex = x, \forall x \in X\) (\(e\) --- нейтральный элемент группы \(G\));
	\item \(g(hx) = (gh)x, \forall g,h \in G, x \in X\). 
\end{enumerate}
\end{definition}

\begin{definition}
\emph{Орбитой} точки \(x \in X\) называется подмножество
\begin{equation*}
	Gx = \{x' \in X \;|\; x' = gx \text{ для некоторого } g \in G\} = \{gx \;|\; g \in G\}
\end{equation*}
\end{definition}

\begin{definition}
\emph{Стабилизатором (стационарной подгруппой)} точки \(x \in X\) называется подгруппа St\((x) = \{g \in G \;|\; gx = x\}\).
\end{definition}

\begin{definition}
Действие \(G\) на \(X\) называется \emph{транзитивным}, если для любых \(x, x' \in X\) найдётся такой элемент \(g \in G\), что \(x' = gx\). Иными словами, все точки множества \(X\) образуют одну орбиту.
\end{definition}

\begin{definition}
Действие \(G\) на \(X\) называется \emph{свободным}, если для любой точки \(x \in X\) условие \(gx = x\) влечёт \(g = e\). Иными словами, St\((x) = \{e\}\) для всех \(x \in X\).
\end{definition}

\begin{definition}
Действие \(G\) на \(X\) называется \emph{эффективным}, если условие \(gx = x\) для всех \(x \in X\) влечёт \(g = e\). Иными словами, \(\bigcup_{x \in X} \mathrm{St}(x) = \{e\}\).
\end{definition}

\begin{definition}
\emph{Ядром неэффективности} действия группы \(G\) на множестве \(X\) называется подгруппа \(K = \{g \in G \;|\; gx = x, \forall x \in X\}\).
\end{definition}

\begin{definition}
Два действия группы \(G\) на множествах \(X\) и \(Y\) называются изоморфными, если существует такая биекция \(\varphi: X \rightarrow Y\), что
\begin{equation*}
	\varphi(gx) = g\varphi(x), \forall g \in G, x \in X.
\end{equation*}
\end{definition}

\begin{proposal}
Всякое свободное транзитивное действие группы \(G\) на множестве \(X\) изоморфно действию группы \(G\) на себе левыми сдвигами.
\end{proposal}

\begin{proposal}
Действия группы \(G\) на себе правыми и левыми сдвигами изоморфны.
\end{proposal}

\begin{theorem}[Кэли]
Всякая конечная группа \(G\) порядка \(n\) изоморфна подгруппе симметрической группы \(S_n\).
\end{theorem}


\section{Лекция 6}

\begin{definition}
\emph{Кольцом} называется множество \(R\) с двумя бинарными операциями <<\(+\)>> (сложение) и <<\(\times\)>> (умножение), обладающими следующими свойствами:
\begin{enumerate}
	\item \((R, +)\) является абелевой группой (называемой \emph{аддитивной группой} кольца \(R\));
	\item выполнены левая и правая дистрибутивности, то есть
	\begin{equation*}
		a(b + c) = ab + ac, \quad (b + c)a = ba + ca, \forall a, b, c \in R;
	\end{equation*}
	\item \(a(bc) = (ab)c, \forall a, b, c \in R\) (ассоциативность умножения);
	\item \(\exists 1 \in R\) (называемый единицей), что
	\begin{equation*}
		a1 = 1a = a, \forall a \in R.
	\end{equation*}
\end{enumerate}
\end{definition}

\begin{definition}
Кольцо \(R\) называется \emph{коммутативным}, если \(ab = ba, \forall a, b \in R\).
\end{definition}

\begin{definition}
Элемент \(a \in R\) называется \emph{обратимым}, если \(\exists b \in R, ab = ba = 1\). 
\end{definition}

\begin{definition}
Элемент \(a \in R\) называется \emph{левым} (соответственно \emph{правым}) \emph{делителем нуля}, если \(a \neq 0\) и \(\exists b \in R, b \neq 0, ab = 0\) (соответственно, \(ba = 0\)).
\end{definition}

\begin{definition}
Элемент \(a \in R\) называется \emph{нильпотентом}, если \(a \neq 0\), и \(\exists m \in \mathbb{N}\), что \(a^m = 0\).
\end{definition}

\begin{definition}
Элемент \(a \in R\) называется \emph{идемпотентом}, если \(a^2 = a\).
\end{definition}

\begin{definition}
\emph{Полем} называется коммутативное кольцо ассоциативное кольцо \(K\) с единицей, в котором всякий ненулевой элемент обратим.
\end{definition}

\begin{proposal}
Кольцо вычетов \(\mathbb{Z}_n\) является полем тогда и только тогда, когда \(n\) --- простое число.
\end{proposal}

\begin{definition}
\emph{Алгеброй} над полем \(K\) (или, кратко, \emph{\(K\)-алгеброй}) называется множество \(A\) c операциями сложения, умножения и умножения на элементы поля \(K\), обладающими следующими свойствами:
\begin{enumerate}
	\item относительно сложения и умножения на элементы из \(K\) множество \(A\) есть векторное пространство;
	\item относительно сложения и умножения \(A\) есть кольцо;
	\item \((\lambda{}a)b = a(\lambda{}b) = \lambda(ab), \forall \lambda \in K; a,b \in A\).
\end{enumerate}

\emph{Размерностью} алгебры \(A\) называется её размерность как векторного пространства над \(K\). Обозначение: \(\mathrm{dim}_K A\).
\end{definition}

\begin{definition}
\emph{Подкольцом} кольца \(R\) называется всякое подмножество \(R' \subset R\), замкнутое относительно операций сложения и умножения (то есть \(a + b \in R', ab \in R', \forall a, b \in R')\) и являющееся кольцом относительно этих операций. \emph{Подполем} называется всякое подкольцо, являющееся полем.
\end{definition}

\begin{definition}
\emph{Подалгеброй} алгебры \(A\) (над полем \(K\)) называется всякое подмножество \(A' \subset A\), замкнутое относительно всех трёх имеющихся в \(A\) операций (сложения, умножения и умножения на элементы из \(K\)) и являющееся алгеброй (над \(K\)) относительно этих операций.
\end{definition}

\begin{definition}
\emph{Изоморфизмом} колец, алгебр называется всякий гомоморфизм, являющийся биекцией.
\end{definition}

\begin{definition}
Подмножество \(I\) кольца \(R\) называется \emph{(двусторонним) идеалом}, если оно является подгруппой по сложению и \(ra \in I, ar \in I, \forall a \in I, r \in R\).
\end{definition}

\begin{definition}
Идеал \(I\) называется \emph{главным}, если существует такой элемент \(a \in R, I = (a)\). В таком случае говорят, что \(I\) порождён элементом \(a\). 
\end{definition}

\begin{definition}
Кольцо \(R/I\) называется \emph{факторкольцом} кольца \(R\) по идеалу \(I\).
\end{definition}

\begin{theorem}[о гомоморфизме для колец]
Пусть \(\varphi: R \rightarrow R'\) --- гомоморфизм колец. Тогда имеет место изоморфизм
\begin{equation*}
	R / \ker \varphi \cong \im \varphi
\end{equation*}
\end{theorem}

\begin{definition}
Кольцо \(R\) называется \emph{простым}, если в нём нет собственных (двусторонних) идеалов.
\end{definition}

\begin{definition}
\emph{Центром} алгебры \(A\) над полем \(K\) называется её подмножество
\begin{equation*}
	Z(A) = \{a \in A \;|\; ab = ba, \forall b \in A\}.
\end{equation*}
\end{definition}

\begin{theorem}
Пусть \(K\) --- поле, \(n\) --- натуральное число и \(A = \mathrm{Mat}(n \times n, K)\) --- алгебра квадратных матриц порядка \(n\) над полем \(K\).
\begin{enumerate}
	\item \(Z(A) = \{\lambda{}E \;|\; \lambda \in K\}\), где \(E\) --- единичная матрица (в частности, \(Z(A)\) --- одномерное подпространство в \(A\));
	\item алгебра \(A\) проста (как кольцо).
\end{enumerate}
\end{theorem}


\section{Лекция 7}

\begin{definition}
Элемент \(b \in R\) \emph{делит} элемент \(a \in R\), (пишут \(b | a\)), если существует элемент \(c \in R, a = bc\).
\end{definition}

\begin{definition}
Два элемента \(a, b \in R\) называют \emph{ассоциированными}, если \(a = bc\) для некоторого обратимого элемента \(c \in R\). 
\end{definition}

\begin{definition}
Кольцо \(R\) без делителей нуля, не являющееся полем, называют \emph{евклидовым}, если существует функция
\begin{equation*}
	N : R \backslash \{0\} \rightarrow \mathbb{Z}_{\geqslant 0},
\end{equation*}
называемая \emph{нормой}, удовлетворяющая следующим условиям:
\begin{enumerate}
	\item \(N(ab) \geqslant N(A), \forall a, b \in R \backslash \{0\}\);
	\item \(\forall a, b \in R, b \neq 0, \exists q, r \in R: a = qb + r\), и либо \(r = 0\), либо \(N(r) < N(b)\).
\end{enumerate}
\end{definition}

\begin{definition}
\emph{Наибольшим общим делителем} элементов \(a\) и \(b\) кольца \(R\) называется их общий делитель, который делится на любой другой их общий делитель. Обозначение: \((a, b)\).
\end{definition}

\begin{theorem}
Пусть \(R\) --- евклидово кольцо и \(a, b\) --- произвольные элементы. Тогда:
\begin{enumerate}
	\item существует наибольший общий делитель \((a, b)\);
	\item существуют такие элементы \(u, v \in R\), что \((a, b) = ua + vb\).
\end{enumerate}
\end{theorem}

\begin{definition}
Кольцо \(R\) называется \emph{кольцом главных идеалов}, если всякий идеал в \(R\) является главным.
\end{definition}

\begin{theorem}
Всякое евклидово кольцо является кольцом главных идеалов.
\end{theorem}

\begin{definition}
Ненулевой необратимый элемент \(p\) кольца \(R\) называется \emph{простым}, если он не может быть представлен в виде \(p = ab\), где \(a, b \in R\) --- необратимые элементы.
\end{definition}

\begin{definition}
Кольцо \(R\) называется \emph{факториальным}, если всякий его ненулевой необратимый элемент <<разложим на простые множители>>, то есть представим в виде произведения (конечного числа) простых элементов, причём это представление единственно с точностью до перестановки множителей и ассоциированности.
\end{definition}

\begin{theorem}
Всякое евклидово кольцо является факториальным.
\end{theorem}

\begin{definition}
Многочлен \(f(x) \in R[x]\) называется \emph{примитивным}, если в \(R\) нет необратимого элемента, который делит все коэффициенты многочлена \(f(x)\).
\end{definition}

\begin{theorem}
Если \(R\) --- факториальное кольцо с полем отношений \(K\) и многочлен \(f(x) \in R[x]\) разлагается в произведение двух многочленов в кольце \(K[x]\), то он разлагается в произведение двух пропорциональных им многочленов в кольце \(R[x]\).
\end{theorem}

\begin{proposal}
Если многчлен \(f(x) \in R[x]\) может быть разложен в произведение двух многочленов меньшей степени в кольце \(K[x]\), то он может быть разложен и в произведение двух многочленов меньшей степени в кольце \(R[x]\).
\end{proposal}

\begin{theorem}
Если кольцо \(R\) факториально, то кольцо многочленов \(R[x]\) тоже факториально.
\end{theorem}

\begin{theorem}
Пусть \(K\) --- произвольное поле. Тогда кольцо многочленов \(K[x_1, \dots, x_n]\) факториально.
\end{theorem}


\section{Лекция 8}

\begin{definition}
Многочлен \(f(x_1, \dots, x_n) \in K[x_1, \dots, x_n]\), где \(K\) --- произвольное поле, называется \emph{симметрическим}, если \(f(x_{\tau(1)}, \dots, x_{\tau(n)}) = f(x_1, \dots, x_n)\) для всякой перестановки \(\tau \in S_n\).
\end{definition}

\begin{definition}
\emph{Элементарными симметрическими многочленами} называются такие многочлены:
\begin{align*}
	\sigma_1(x_1, \dots, x_n) &= x_1 + x_2 + \dots + x_n; \\
	\sigma_2(x_1, \dots, x_n) &= \sum_{1 \leqslant i < j \leqslant n} x_ix_j; \\
	\dots &= \dots \\
	\sigma_k(x_1, \dots, x_n) &= \sum_{1 \leqslant i_1 < i_2 < \dots < i_k \leqslant n} x_{i_1}x_{i_2}\dots{}x_{i_k}; \\
	\dots &= \dots \\
	\sigma_n(x_1, \dots, x_n) &= x_1x_2\dots{}x_n;
\end{align*} 
\end{definition}

\begin{theorem}[Основная теорема о симметрических многочленах]
Для всякого симметрического многочлена \(f(x_1, \dots, x_n)\) существует и единственен такой многочлен \(F(y_1, \dots, y_n)\), что
\begin{equation*}
	f(x_1, \dots, x_n) = F(\sigma_1(x_1, \dots, x_n), \dots, \sigma_n(x_1, \dots, x_n)).
\end{equation*}
\end{theorem}

\begin{definition}
\emph{Старшим членом} ненулевого многочлена \(f(x_1, \dots, x_n)\) называется наибольший в лексикографическом порядке встречающийся в нём одночлен. Обозначение: \(L(f)\).
\end{definition}

\begin{theorem}[о старшем члене]
Пусть \(f(x_1, \dots, x_n), g(x_1, \dots, x_n) \in K[x_1, \dots, x_n]\) --- произвольные ненулевые многочлены. Тогда \(L(fg) = L(f)L(g)\).
\end{theorem}

\begin{theorem}[Виет]
Пусть \(\alpha_1, \dots, \alpha_n\) --- корни многочлена \(x^n + a_{n-1}x^{n-1} + \dots + a_1x + a_0\). Тогда
\begin{equation*}
	\sigma_k(\alpha_1, \dots, \alpha_n) = (-1)^k a_{n-k}, \quad k = 1, \dots, n.
\end{equation*}
\end{theorem}

\begin{definition}
\emph{Дискриминантом} многочлена \(h(x) = a_nx^n + \dots + a_1x + a_0\) с корнями \(\alpha_1, \dots, \alpha_n\) называется выражение
\begin{equation*}
	D(h) = a_{n}^{2n-2} \prod_{1 \leqslant i < j \leqslant n} (\alpha_i - \alpha_j)^2
\end{equation*}
\end{definition}


\section{Лекция 9}

\begin{definition}
Пусть \(K\) --- произвольное поле. \emph{Характеристикой} поля \(K\) называется такое наименьшее натуральное число \(p\), что \(\underbrace{1 + \dots + 1}_p = 0\). Если такого числа не существует, то говорят, что характеристика поля равна нулю. Обозначение: char \(K\).
\end{definition}

\begin{proposal}
Характеристика произвольного поля \(K\) либо равна нулю, либо является простым числом.
\end{proposal}

\begin{definition}
Пересечение любого семейства подполей фиксированного поля \(K\) является подполем в \(K\). В частности, для всякого подмножества \(S \subseteq K\) существует наименьшее по включению подполе в \(K\), содержащее \(S\). Это подполе совпадает с пересечением всех подполей в \(K\), содержащих \(S\). Из этого следует, что в каждом поле существует наименьшее по включению подполе, оно называется \emph{простым подполем}.
\end{definition}

\begin{proposal}
Пусть \(K\) --- поле и \(K_0\) --- его простое подполе. Тогда:
\begin{enumerate}
	\item если \(\mathrm{char }K = p > 0\), то \(K_0 \cong \mathbb{Z}_p\);
	\item если \(\mathrm{char }K = 0\), то \(K_0 \cong \mathbb{Q}\).
\end{enumerate}
\end{proposal}

\begin{definition}
Если \(K\) --- подполе \(F\), то говорят, что \(F\) --- \emph{расширение} поля \(K\).
\end{definition}

\begin{definition}
\emph{Степенью} расширения полей \(K \subseteq F\) называется размерность поля \(F\) как векторного пространства над полем \(K\). Обозначение: \([F : K]\).
\end{definition}

\begin{definition}
Расширение полей называется \emph{конечным}, если \([F : K] < \infty\). 
\end{definition}

\begin{proposal}
Пусть \(K \subseteq F\) и \(F \subseteq L\) --- конечные расширения полей. Тогда расширение \(F \subseteq L\) тоже конечно и \([L : K] = [L : F][F : K]\).
\end{proposal}

\begin{definition}
Элемент \(\alpha \in F\) называется \emph{алгебраическим} над подполем \(K\), если существует ненулевой многочлен \(f(x) \in K[x]\), для которого \(f(\alpha) = 0\). В противном случае \(\alpha\) называется \emph{трансцедентным} элементом над \(K\). 
\end{definition}

\begin{definition}
\emph{Минимальным многочленом} алгебраического элемента \(\alpha \in F\) над подполем \(K\) называется ненулевой многочлен \(h_\alpha (x)\) наименьшей степени, для которого \(h_\alpha (\alpha) = 0\).
\end{definition}

\begin{proposal}
Пусть \(\alpha \in F\) --- алгебраический элемент над \(K\) и \(n\) --- степень его минимального многочлена над \(K\). Тогда
\begin{equation*}
	K(\alpha) = \{\beta_0 + \beta_1 \alpha + \dots + \beta_{n-1}\alpha^{n-1} \;|\; \beta_0, \dots, \beta_{n-1} \in K\}.
\end{equation*}
Кроме того, элементы \(1, \alpha, \alpha^2, \dots, \alpha^{n-1}\) линейно независимы над \(K\). В частности, \([K(\alpha) : K] = n\).
\end{proposal}

\begin{theorem}
Пусть \(K\) --- произвольное поле и \(f(x) \in K[x]\) --- многочлен положительной степени. Тогда существует конечное расширение \(K \subseteq F\), в котором многочлен \(f(x)\) имеет корень.
\end{theorem}

\begin{definition}
Пусть \(K\) --- некоторое поле и \(f(x) \in K[x]\) --- многочлен положительной степени. \emph{Полем разложения} многочлена \(f(x)\) называется такое расширение \(F\) поля \(K\), что:
\begin{enumerate}
	\item многочлен \(f(x)\) разлагается на линейные множители;
	\item корни многочлена \(f(x)\) не лежат ни в каком собственном подполе поля \(F\), содержащем \(K\).
\end{enumerate}
\end{definition}

\begin{theorem}
Поле разложения любого многочлена \(f(x) \in K[x]\) существует и единственно с точностью до изоморфизма, тождественного на \(K\).
\end{theorem}


\end{document}