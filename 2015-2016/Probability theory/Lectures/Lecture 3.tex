\documentclass[a4paper, 12pt]{article}
\usepackage[english,russian]{babel}
\usepackage[utf8]{inputenc}
\usepackage[left=2cm,right=2cm,top=2cm,bottom=2cm]{geometry}
\usepackage{amsmath}
\usepackage{amssymb}
\usepackage{amsthm}
\usepackage{amsfonts}
\usepackage{mathtools}
\usepackage{relsize}
\usepackage{microtype}
\usepackage{graphicx}
\usepackage{url}


\renewcommand{\qedsymbol}{$\blacksquare$}

\everymath{\displaystyle}

\theoremstyle{definition}
\newtheorem{Definition}{Определение}
\newtheorem{Lemma}{Лемма}
\newtheorem{Examples}{Примеры}
\newtheorem{Consequence}{Следствие}
\newtheorem{Thm}{Теорема}
\newtheorem{Note}{Замечание}

\title{Теория вероятностей и математическая статистика.\\Лекция 3.
    \footnote{Свёрстано Жуковым Иваном. Материал может содержать ошибки-хуешибки}}
\author{}
\date{16 сентября 2015}
\begin{document}
    \maketitle{}

    \begin{Definition}
        События \(A_1, \dots, A_n\) называются \textbf{независимыми в совокупности}, если для всех индексов
        \( \underset{k = 2 \ldots n}{ 1 \le i_1 \le \ldots \le i_k \le n:}\)

        \begin{equation}{\label{eq:1}}
            P(A_{i_1} \dots A_{i_n}) = P(A_{i_1}) \cdot \ldots \cdot P(A_{i_n}) \tag{*}
        \end{equation}

        Если \(\eqref{eq:1}\) выполняется для \(k = 2\), то события называются \textbf{независимыми попарно}.
    \end{Definition}

    \begin{center}
        \textit{Элементы теории надёжности}.
    \end{center}

    \begin{Definition}
        Надёжностью системы называется вероятность её безотказной работы.
    \end{Definition}

    \begin{Definition}
        \textbf{Схемой Испытаний Бернулли (СИБ)} называется модель по проведению \(n\) испытаний, удовлетворяющих трём условиям:
            \begin{enumerate}
                \item
                    В каждом испытании возможно 2 исхода: \(A\  \text{и} \ \overline{A} \)
                \item
                    Все испытания независимы.
                \item
                    Значение \(P(A) = P\) не меняется во всех опытах. 
            \end{enumerate}

            Представим простой эксперимент в котором возможно два исхода: удача и неудача. Соответственно, вероятности удачи и неудачи обозначим за \(0 < p < 1\) и \(q = 1 - p\).

            Теперь рассмотрим новый эксперимент, который заключается в \(n\)-кратном повторении эксперимента предыдущего.
            \begin{itemize}
                \item
                    Тогда пространство элементарных событий можно представить как \(\Omega = (a_1, \ldots, a_n)\), где каждое элементарное событие \(a_i\) может принимать значения 1 или 0, в зависимости от успеха.
                \item
                    Пусть \(k\) - количество удачных опытов из n. Нетрудно заметить, что \(k = \sum^{n}_{1} a_i\).
                \item
                    Теперь каждому событию (обозначим его просто за \(\omega\)) сопоставим число --- вероятность успеха: \(p(\omega) = p^{\sum^{n}_{1} a_i} \cdot q^{n - {\sum^{n}_{1} a_i}} = p^k \cdot q^{n - k} \)
                \item
                    Пусть \(A_k = \{\omega| \sum^{n}_{1} a_i = k\} \). Тогда выполняется Теорема 1.
            \end{itemize}
    \end{Definition}

    \newpage

    \begin{Thm}
        \textbf{Формула Бернулли}: \(P(A_k) = C^{k}_{n} \cdot p^k \cdot (1 - p)^{n - k}\)
    \end{Thm}

    \begin{proof}
        \leavevmode
        \begin{itemize}
            \item
                \(A_k = \) \{в \(n\) испытаниях \(k\) успехов\} \(= \{\omega| \sum^{n}_{1} a_i = k\}\);
            \item
                \(P(\omega) = p^k \cdot (1 - p)^{n - k}\)
            \item
                \(A_k = \sum_{\omega \in A_k} \omega \Rightarrow  P(A_k) = P(\sum_{\omega \in A_k}  \omega) =
                \left| \text{события \(\omega\) несовместны} \right| = \sum_{\omega \in A_k} P(\omega) =\\
                = C^{k}_{n} \cdot p^n \cdot (1 - p)^{n - k}\) 
        \end{itemize}
    \end{proof}

    \begin{Note}
        \(C^{k}_{n}\) --- из \(n\) опытов выбираем \(k\) удачных.
    \end{Note}

    \begin{Definition}
        \textbf{Полиномиальная схема \footnote{материал взят с Вики}}.

        Обычная формула Бернули применима на случай, когда при каждом испытании возможно одно из двух событий. Формулу Бернулли
        можно обобщить на случай, когда при каждом испытании происходит одно и только одно из  \(k > 2\)  событий с вероятностью
        \(\underset{i = 1, 2, \ldots, k}{p_i}\), где \(\sum^{k}_{1} p_i = 1\) . Вероятность появления  \(m_1\)  раз первого события
        и  \(m_2\)  - второго и  \(m_k\)  раз k-го находится по формуле

        \[P_n(m_1,m_2,...,m_k)= \frac{n!}{m_1! m_2! \ldots m_k!} \cdot {p_1}^{m_1} \cdot {p_2}^{m_2} \cdot \ldots \cdot {p_k}^{m_k}\]
        , где  \(n = m_1 + m_2 + \ldots +m_k\). 
    \end{Definition}

    \begin{Definition}
        Случайные события \(H_1, \ldots, H_n \subset \Omega\) называются \textbf{полной группой событий}, если:
            \begin{enumerate}
                \item
                    \(\underset{\text{несовместные}}{H_i \cdot H_j} = \varnothing\) при \(i \neq j\)
                \item
                    \(\sum^{n}_{1} H_i = \Omega\)
            \end{enumerate}
    \end{Definition}

    \begin{Thm}
        Пусть \(H_1, \ldots, H_n\) - полная группа событий и \(A \subset \Omega\). Тогда:
        \[P(A) = \sum^{n}_{1} P(H_i) \cdot P(A / H_i)\]
    \end{Thm}

    \begin{proof}
        \(P(A) = P(A \cdot \Omega) = P(A \cdot \sum^{n}_{1} H_i) = P(\underset{\subseteq H_1}{AH_1}
        + \ldots + \underset{\subseteq H_n}{AH_n}) = \sum^{n}_{1}P(AH_i) =
        \left| \text{т.к. \(H_1, \ldots, H_n\) несовместны} \right| = \sum^{n}_{1} P(H_i) \cdot P(A / H_i)\)
    \end{proof}
 \end{document}